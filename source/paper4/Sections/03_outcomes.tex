\section{Results and Discussion}


In 2023, we conducted a six-month CS Capstone course at UCI where we piloted the use of the SPMT. We then evaluated the effectiveness of the SPMT by gathering feedback from both students and sponsors. We compared this feedback to that gathered in the 2022 course taught by the same instructor when students were placed in teams based solely on their project and team member preferences. Results show that the SPMT maximized the opportunity for students to learn new skills since it considered student interests in the matching process, and it increased student satisfaction with their groups and projects. The SPMT not only helped to match students with better educational projects but also acted as a catalyst in boosting student-project ownership and a deeper understanding of the SWE profession. This also led to higher sponsor satisfaction with the group formation and project matching during the piloting period.




\subsection{Sponsor Feedback}
Sponsor feedback during the pilot was compared to feedback during the previous year before the SPMT was introduced. At the end of the first term in both offerings, sponsors had the opportunity to review their team's performance.

In the year before the SPMT, two out of the six project reviews expressed discontent with student progress and preparedness. Industry sponsors stated that ``the students' readiness to do hands-on programming was less than expected'' and ``they veered off of the original requirements which led to a gap between what they developed and what was expected''. Additionally, only one sponsor expressed satisfaction working with their team.


The benefits of the switch to SPMT in 2023 were evident in the shift of general sentiment to much more positive reviews during the feedback stage. Out of nine submitted team reviews, four sponsors recognized their team's efforts by sharing that the teams are ``really scrappy'', ``doing a great job'', ``show high levels of enthusiasm'', and ``quick learners''. In contrast, only one comment pointed to areas of improvement, saying that ``communication could be better'', which was not one of the skills targeted by SPMT.
The shift to primarily positive feedback from sponsors shows the teams' ability to meet their sponsor's expectations, highlighting the value of the team matching tool.


\subsection{Qualitative Student Feedback}
Student course evaluations provided a means of understanding student experiences before and after the introduction of the Student-Project Matching Tool. 

In 2022, 15 out of 29 students filled out the survey. One question asked students to mention any aspects of the course that could be modified to improve their learning. Of the 15 responses, three mentioned that team and project matching could be improved. 

Overall, feedback pointed to the common weaknesses encountered when only considering project and team member preferences in the matching process.

In 2023, 43 out of 47 students filled out the survey. The total number of responses almost tripled due to the larger class size and higher response rate. Despite a higher number of participants, only two responses to the improvements question mentioned the lack of technical knowledge for the project. This type of comment was expected because the tool was designed to challenge students to acquire new skills that they showed interest in.





