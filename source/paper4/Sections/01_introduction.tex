\section{Introduction}


For many decades, researchers and instructors have aimed to optimize team formation in student engineering projects. Brickell et al found that allowing for the self-formation of teams led to negative student attitudes towards their courses, instructors, projects, and more~\cite{brickell_assigning_1994}. With justification for instructor-created teams established, the focus has shifted towards optimizing the team formation process. To this end, Layton et al wrote a digital tool in an attempt to computationally pick the ``best'' teams by asking students questions and weighting their responses \cite{layton_design_2010}. However, these approaches' matching strategies are solely dependent on the variables each instructor or researcher chooses to incorporate in their algorithm, not taking into account project needs directly. For instance, Smyser and Jaeger found that a key factor of success in capstone teams is in students' passion and ownership of their project~\cite{smyser_how_2015}. 
Therefore, it is crucial that we not only optimize the matching of students into teams, but also the matching of students to projects, ensuring that all project skills requirements can be fulfilled.

Since Conn and Sharpe in 1993~\cite{conn_industry-sponsored_1993}, it has become commonplace for many capstone courses to match student teams to industry-sponsored projects. However, this introduces another layer of complexity to the team-formation and matching process, as industry stakeholders have their own set of team requirements. Thus, any computational team-forming tool must not only optimize student-centered variables but also stakeholder-centered variables.

\subsection{Goals}

The primary goal of this study was to design and test the Student-Project Matching Tool (SPMT) for a CS Software Engineering (SWE) Capstone course. This innovative platform is designed to aid instructors in forming student teams and matching them with industry-sponsored projects to improve student outcomes. Specifically, we designed the SPMT to support the following outcomes: (i) increased student-project buy-in/ownership, (ii) broadened student skillsets, and (iii) fulfilled project skills requirements.


We tested our desired outcomes by piloting the SPMT in a six-month-long CS SWE Capstone course at the University of California, Irvine (UCI) during the 2022-2023 academic year to form student groups and assign them to the best industry-sponsored project match. The SPMT determined the best match by considering students' interests and skills in SWE topics relevant to each project, and conversely ensuring that the skills across all students in a group collectively met the needs of the project they were matched to.


