\documentclass{article}

\usepackage[
  paperheight=8.5in,
  paperwidth=5.5in,
  left=10mm,
  right=10mm,
  top=20mm,
  bottom=20mm]{geometry}
\usepackage[utf8]{inputenc}

\usepackage{graphicx}
\usepackage{wrapfig}
\usepackage[bottom]{footmisc}
\usepackage{listings}
\usepackage{enumitem}

\usepackage{wrapfig}
\usepackage{ragged2e}

\usepackage{array}
\usepackage[table]{xcolor}
\usepackage{multirow}
\usepackage{booktabs}
\usepackage{hhline}
\definecolor{palegreen}{rgb}{0.6,0.98,0.6}

\usepackage{amsmath}
\usepackage{amssymb}
\usepackage{multicol}
\usepackage{lipsum}
\usepackage{hyphenat}
\PassOptionsToPackage{hyphens}{url}
\usepackage{url}

\usepackage{rotating}

%\usepackage{xeCJK}

%% support use of straight quotes in code listings
\usepackage[T1]{fontenc}
\usepackage{textcomp}
\usepackage{listings}
\lstset{upquote=true}

%% for shrinking space between lines
\usepackage{setspace}

\newcommand*{\affaddr}[1]{#1} % No op here. Customize it for different styles.
\newcommand*{\affmark}[1][*]{\textsuperscript{#1}}
\newcommand*{\email}[1]{\small{\texttt{#1}}}
\newcommand{\tarot}{\textsc{Tarot}}
\renewcommand*\contentsname{\centering Table of Contents}

\renewcommand{\footnoterule}{%
  \kern -3pt
  \hrule width \textwidth height 0.5pt
  \kern 2pt
}

% remove date
\date{}

\usepackage{titlesec}
\titleformat*{\section}{\large\bfseries}
\titleformat*{\subsection}{\normalsize\bfseries}
\titleformat*{\subsubsection}{\normalsize\bfseries}

\usepackage{biblatex}


\usepackage{algorithm}
\usepackage{algpseudocode}

\addbibresource{references.bib}

\title{Using Participatory Narrative Inquiry to Explore Cooperative Education in Computing Education\footnote{\protectCopyright \copyright 2024 by the Consortium for Computing Sciences in Colleges.
Permission to copy without fee all or part of this material is granted provided
that the copies are not made or distributed for direct commercial advantage,
the CCSC copyright notice and the title of the publication and its date appear,
and notice is given that copying is by permission of the Consortium for
Computing Sciences in Colleges.  To copy otherwise, or to republish, requires
a fee and/or specific permission.
}
}

\author{
  Sebastian Dziallas\\
  Department of Computer Science\\
  University of the Pacific\\
  Stockton, CA 95211\\
  \email{sdziallas@pacific.edu}
}

\begin{document}
\maketitle

\thispagestyle{empty}
\pagestyle{empty}

\begin{abstract}
This paper reports on an exploratory study using a novel methodological approach, Participatory Narrative Inquiry, to explore computing students’ experiences in a cooperative education (co-op) program. This work was conducted at a medium size, private, student-centered university in the Western United States, where many computer science students take part in a co-op program. We conducted interviews with five students who had participated in a co-op and report on themes that emerged from the data. The initial findings suggest that there are opportunities for future work using Participatory Narrative Inquiry to study cooperative education and students’ educational experiences more broadly.

\end{abstract}

\section{Introduction}
Work-based learning aims to integrate academic learning with real-world experiences \cite{fincherPorousClassroomProfessional2013a, maguireBackFutureShaping2019}. Across the world, different models have emerged, such as the placement years and degree-level apprenticeships in the UK and the “Duales Studium” in Germany. In the US, work-based learning commonly takes the form of internships and co-op programs, which involve students working for an employer in paid positions at one or multiple points during their education. Its benefits, especially in terms of career outcomes, have been widely reported in the literature \cite{ramirezAcademicOutcomesCooperative2015}. Several institutions, such as Northeastern University and Drexel University, have incorporated cooperative education into their curricula. At the university where this study was conducted, students taking part in the co-op program typically work for an employer for two terms (e.g. the summer and fall semester following their junior year).

In this work, we use a narrative approach to explore students’ experiences in the co-op program. Narrative methods have previously been used to study students’ and graduates’ educational experiences \cite{dziallasAspectsGraduatenessComputing2016, holmegaardWhereEngineeringApplied2016}. Research has shown that we construct stories to make sense of our lives \cite{mcadamsWhatWeKnow1995}, so narrative methods – which aim to elicit stories from participants – are an especially appropriate way of exploring lived experiences. We use an approach called Participatory Narrative Inquiry (PNI). To our knowledge, this approach has not been used in computing education research to date, but is similar to the SenseMaker tool that has been introduced in engineering education research \cite{sochackaUsingSenseMakerExamine2020}.

Participatory Narrative Inquiry combines aspects of mixed-methods research and oral history approaches. As one of the founders of PNI observes: “PNI is an approach in which groups of people participate in gathering and working with raw stories of personal experience to make sense of complex situations for better decision making. PNI focuses on the profound consideration of values, beliefs, feelings, and perspectives through the recounting and interpretation of lived experience.” \cite{kurtzWorkingStoriesYour2014} We chose Participatory Narrative Inquiry because of its ability to elicit stories from a larger number of participants and to engage participants actively in the research process.
	
This work aims to address the following two research questions:

\begin{itemize}
    \item How can Participatory Narrative Inquiry be used to investigate students’ experiences in computing education?
    \item How do students make sense of their experiences in a cooperative education program?
\end{itemize}

\section{Methodology}
We obtained ethics approval and contacted all 99 computer science students at a medium size, private, student-centered university who had completed a co-op since spring 2019 with an invitation to participate in this study and to share their experiences in the co-op program. (There was a lower number of students who took part in a co-op in 2020 and 2021 due to the COVID-19 pandemic.) The overall response rate was low, likely because the co-op office at the university only retains students’ university email addresses, which are deactivated after graduation.

We ultimately conducted individual interviews with 5 students. Three of these students were in their senior year and two had already graduated. The interviews lasted between 30 minutes and 1 hour and were audio recorded, transcribed, and anonymized. During the interviews, participants were asked to choose from and respond to one of the following prompts, which were specifically designed to elicit stories.

\begin{itemize}
    \item Describe the moment when you decided to participate in the co-op program. What went through your mind?
    \item What was the highest or lowest moment of your experience in the co-op program? What happened in that moment?
    \item Looking back over your experiences in co-op, what one moment stands out to you in terms of what came later?
    \item Can you recall a time during your experience in the co-op program that will stay with you for a long time - for any reason, good or bad? What happened that you will remember?
    \item If none of these questions appeal to you, please choose any experience you had during the co-op program - good or bad - that you would like to tell us about. What happened that mattered to you?
\end{itemize}

After responding to the prompt of their choice, participants were also invited to revisit the list of questions and to respond to additional prompts, which all of them chose to do.

As part of the analysis, we initially identified the stories participants told. Mattingly argues that stories “are about someone trying to do something, and what happens to her and to others as a result” \cite{mattinglyHealingDramasClinical1998}. We thus used indicators, like personal pronouns, past-tense verbs, and time references to identify stories in the transcripts \cite{kurtzWorkingStoriesYour2014}. We found a total of 18 stories.

Participatory Narrative Inquiry typically involves participants themselves making sense of their stories. However, none of the interviewees responded to an invitation to participate in such a “sensemaking session”. We invited several volunteers who were familiar with Participatory Narrative Inquiry to take part in such a session instead. As part of this session, participants completed the following steps:

\begin{itemize}
    \item Read through the stories and gave each story a title
    \item Arranged stories on a temporal axis (i.e. before, during, or after co-op)
    \item Considered whether or not a story reflected a successful co-op experience
    \item Discussed any emerging themes from the analysis
\end{itemize}

In the following, we highlight initial findings from this work.

\section{Findings}
As part of the sensemaking session, participants identified several themes in the data. One of these themes was related to students’ experiences applying for co-op positions. At the university, students planning to take a co-op are required to complete a one-credit course designed to help them prepare their resume and to connect them with employers the university has established relationships with.

% Maybe attribute quotes

\begin{quote}
    \textit{I'd say the hardest part for me was just getting my foot in the door and actually getting the co-op itself, because once you're in the co-op, assuming it's the right fit, you'll learn no matter what. You'll pick it up. You'll mess up. You'll learn. You'll do cool things. So you'll have a great time. But actually landing the co-op itself was really difficult for me. It was kind of stressful, especially for my semester. I heard that we had a whole lot of computer science students and not a lot of openings. So there was the big name [company] and I didn't get that. I'm like, “Well, I guess I'm doomed.” Because all of the other ones […] sounded like I.T. jobs or just general engineering jobs, not specifically focused for computer science. So then my other option was applying outside of [the university] through LinkedIn and that sort of thing. But it's way harder to get a response there.} (Student 1)
\end{quote}

Some students also sought out co-op positions outside of the university’s relationships with employers. One student attended a conference where he spoke with a recruiter and was able secure an interview and ultimately a position in this way.

\begin{quote}
    \textit{The reason I went to the conference in the first place was because I didn’t have anything else. I had had an interview with [large software company] and I couldn't get any interviews anywhere else. I applied to at least over 75 positions. It was a point where I'm like, “OK, I know I'm putting the work in, but nothing's coming from it.” So, you know, to go to the conference, get the position that I wanted, and looking back on it now I'm glad that the other stuff didn't work out so I could be where I am today.} (Student 2)
\end{quote}

One question that emerged in the sensemaking session was about how the co-op students fit into their workplace. While this depends in part on the workplace environment, the interview participants reported positive experiences.

\begin{quote}
    \textit{I think it's the people that still stand out to me and that I still think about. They were very involved in our professional development. The IT director would come and talk to us and take us to meetings and teach us about financial developments – not only relevant to the work that we're doing, but he was also supporting our growth outside of the stuff we're doing. So I think they were involved and if we didn't know something, they would make us sit down and teach us all the concepts from scratch, even if it took an hour or two. They were always super helpful.} (Student 5)
\end{quote}

\begin{quote}
    \textit{The highest moment in my co-op was when was tasked with presenting to the VPs of the whole operation and facilities. So it's me, I'm just an intern, and I was presenting in this boardroom to the VPs about my project. I would say that was the highest moment, because, you know, I'm just 22 and they're all in their 50s and 60s. I put on a suit and they said it was a very good presentation.} (Student 4)
\end{quote}

The second quote is related to a theme of accomplishments. Students discussed how a sense of autonomy during the co-op helped them build their confidence.

% So before going into the co-op, I had absolutely no idea about cyber security. I was just put into it. [...] But during the course, I've learned a lot. But 

\begin{quote}
    \textit{What really helped was the independent projects that I was able to take, where they said, “it's up to you, you can do it or not.” So I was able to do some programming and automate some stuff. So I felt very accomplished. And I think that's one of the things that made me want to continue and I still do that to this day. That's basically helped with my confidence, because I was successful once and now I have that confidence in myself that I can do it again.} (Student 5)
\end{quote}

\begin{quote}
    \textit{My highest moment for it was when I actually got to make my own script, because for the most part, I was just fixing bugs, working on documentation, that kind of thing. But I was assigned an actual task of creating a little Python script [...]. And it took me a long time to get it. But I was really proud of that because that was my contribution to the project. I did it all by myself. And it was it was really nice seeing that I could actually do it, especially when the people were getting back to me like, “hey, this actually works. Thanks for making it.”} (Student 1)
\end{quote}

The experiences described by this student are an example of \textit{legitimate peripheral participation} \cite{laveSituatedLearningLegitimate1991}. While “fixing bugs” and “working on documentation” may seem like minor contributions, they are nonetheless valuable. This matches prior findings in the literature on work-based learning in the UK \cite{dziallasLearningContextFirst2021a}.

Finally, the participants in the sensemaking session also identified an opportunity related to students’ transitions to and from their co-op. At this university, students are required to return for (at least) one semester after their co-op. Several students reflected on this return to the university:

% Because at the co-op, I was sitting in a box all day. I'm not coding too much. So I kind of wanted to go back to education where I felt like I’m learning more. I was learning during my co-op, but it wasn't through lectures. It was more hands on. And now that I'm back in education, I wish I could just go back to working like nine to five – once you clock out, you're totally free. It's definitely a different experience.

\begin{quote}
    \textit{I hear that for a lot of other students, they take the co-op up and then they come back to the university and they're like, “Why am I doing this? What's the point? I have another job.” But, personally, it's OK. It's kind of like when you're at one, you're wishing for the other and then vice versa.} (Student 1)
\end{quote}

\begin{quote}
    \textit{When I did have to come back, it was kind of strange, because I'd been working full time for six months. [...] It was just a weird transition from working every day, waking up at six and then driving to the city, going inside the office, checking in. And then all of a sudden I can wake up anytime I want. I could do my homework anytime I want. It was freeing and it was also just so different. It took a while to get used to.} (Student 3)
\end{quote}

% In that study, students reflected on the importance of their learning experiences at university in the workforce and returned with improved study habits.

This differs from prior findings, such as in the UK, where students reported approaching their final year at university differently after spending a year in industry \cite{dziallasAspectsGraduatenessComputing2016}. Indeed, for many of the students in that study, their work-based learning experience served as a turning point. Future work may explore this transition to the university as well as differences among work-based learning models further.

\section{Conclusions}
This paper makes two main contributions. First, it contributes a study in computing education using a novel methodological approach, Participatory Narrative Inquiry. Second, it offers a narrative perspective on students’ experience in a cooperative education program and identifies aspects that educators at other institutions may consider. As this was preliminary study with a small number of participants from a single institution, the findings presented here are naturally limited. However, there are opportunities for future work to use Participatory Narrative Inquiry to further explore students’ experiences in co-op programs and in computing education more broadly.

% More students and more sensemaking sessions.

\clearpage

\renewcommand*{\bibfont}{\fontsize{10}{10}\selectfont}
\printbibliography

\end{document}
